\documentclass[12pt,a4paper]{report}

% ================== GÓI TIẾNG VIỆT & CƠ BẢN ==================
\usepackage[utf8]{inputenc}
\usepackage[utf8]{vntex}
\usepackage[english,vietnamese]{babel}

% ================== GÓI HỖ TRỢ ĐỊNH DẠNG ==================
\usepackage{geometry}
\geometry{
    a4paper,
    left=20mm,
    right=20mm,
    top=20mm,
    bottom=25mm
}
\usepackage{setspace}
\onehalfspacing % Dãn dòng 1.5
\usepackage{indentfirst} % Thụt đầu dòng đoạn đầu tiên
\usepackage{float}       % Định vị hình ảnh [H]
\usepackage[dvipsnames]{xcolor} % Màu sắc
\usepackage{graphicx}    % Chèn ảnh
\usepackage{array}       % Hỗ trợ bảng
\usepackage{tabularx}
\usepackage{booktabs}
\usepackage{multirow}
\usepackage{multicol}
\usepackage{subcaption}  % Hình phụ

% ================== TRANG TRÍ & VẼ HÌNH (CHO TRANG BÌA) ==================
\usepackage{tikz}
\usetikzlibrary{arrows,snakes,backgrounds,calc}

% ================== HEADER / FOOTER (CHUẨN BÁCH KHOA) ==================
\usepackage{fancyhdr}
\setlength{\headheight}{40pt}
\pagestyle{fancy}
\fancyhead{} % Xóa header mặc định
\fancyfoot{} % Xóa footer mặc định

% Footer trái: Tên Khoa/Trường
\fancyfoot[L]{\scriptsize \ttfamily Khoa Khoa Học và Kỹ Thuật Máy Tính}
% Footer phải: Số trang
\fancyfoot[R]{\scriptsize \ttfamily Trang {\thepage}}

\renewcommand{\headrulewidth}{0pt}   % Không kẻ vạch trên header
\renewcommand{\footrulewidth}{0.3pt} % Kẻ vạch dưới footer

% ================== CODE LISTINGS (TRÌNH BÀY CODE ĐẸP) ==================
\usepackage{listings}
\definecolor{codebg}{RGB}{245,245,245}
\definecolor{codeframe}{RGB}{200,200,200}
\lstdefinestyle{report}{
    backgroundcolor=\color{codebg},
    frame=single,
    rulecolor=\color{codeframe},
    basicstyle=\ttfamily\small,
    keywordstyle=\color{blue}\bfseries,
    commentstyle=\color{green!60!black},
    stringstyle=\color{red},
    breaklines=true,
    captionpos=b,
    tabsize=2,
    numbers=left,
    numberstyle=\tiny\color{gray}
}
\lstset{style=report}

% ================== TÀI LIỆU THAM KHẢO (BIBLATEX) ==================
\usepackage[
    backend=biber,
    style=numeric,
    sorting=none,
    language=english
]{biblatex}
\addbibresource{refs.bib} % Đảm bảo bạn có file refs.bib

% ================== TIÊU ĐỀ CHƯƠNG (FORMAT LẠI CHO GỌN) ==================
\usepackage{titlesec}
\titleformat{\chapter}[hang]{\bfseries\LARGE\uppercase}{\thechapter.}{1em}{}
\titlespacing*{\chapter}{0pt}{-20pt}{20pt}

% ================== LIÊN KẾT (HYPERREF) ==================
\usepackage[unicode,hidelinks]{hyperref}
\hypersetup{
    colorlinks=false,
    pdfborder={0 0 0}
}

% Đổi tên các thành phần sang tiếng Việt (nếu babel chưa tự đổi hết)
\renewcommand{\contentsname}{Mục lục}
\renewcommand{\listfigurename}{Danh sách hình vẽ}
\renewcommand{\listtablename}{Danh sách bảng}
\renewcommand{\bibname}{Tài liệu tham khảo}

%%%%%%%%%%%%%%%%%%%%%%%%%%%%%%%%%%%%%%%%%%%%%%%%%%%%%%%%%%%%%%%%%%%%%%%%%
\begin{document}

% ===================== TRANG BÌA =====================
% Gọi file trang bìa riêng để code gọn gàng
\begin{titlepage}
    % --- VẼ KHUNG VIỀN (BORDER) ---
    \begin{tikzpicture}[remember picture, overlay]
        \draw [line width=3pt]
            ($(current page.north west) + (3.0cm,-2.0cm)$)
            rectangle
            ($(current page.south east) + (-2.0cm,2.0cm)$);
        \draw [line width=1pt]
            ($(current page.north west) + (3.15cm,-2.15cm)$)
            rectangle
            ($(current page.south east) + (-2.15cm,2.15cm)$);
    \end{tikzpicture}

    \begin{center}
        \vspace{-0.5cm}
        \textbf{\large ĐẠI HỌC QUỐC GIA TP. HỒ CHÍ MINH}\\
        \textbf{\large TRƯỜNG ĐẠI HỌC BÁCH KHOA}\\
        \textbf{\large KHOA KHOA HỌC VÀ KỸ THUẬT MÁY TÍNH}
        
        \vspace{2.5cm}
        
        % Logo (Nhớ file hcmut.png phải có trong thư mục img/)
        \includegraphics[width=0.35\textwidth]{img/hcmut.png} 
        
        \vspace{2.5cm}
        
        {\Large \bfseries BÁO CÁO ĐỒ ÁN MÔN HỌC} \\
        \vspace{0.5cm}
        {\huge \bfseries ỨNG DỤNG CHIA SẺ ẨN DANH \\ \& CHỮA LÀNH TÂM HỒN} \\
        \vspace{0.5cm}
        {\huge \bfseries SEAWHISPER}
        
        \vspace{3cm}
        
        \begin{table}[h]
            \centering
            \begin{tabular}{l l}
                \textbf{GVHD:} & Hoàng Lê Hải Thanh \\
                 & \\
                \textbf{Nhóm thực hiện:} & SeaWhisper Team \\
            \end{tabular}
        \end{table}
        
        \vfill
        
        \textbf{TP. HỒ CHÍ MINH, THÁNG 01/2026}
        
    \end{center}
\end{titlepage} 

% ===================== PHẦN ĐẦU (FRONT MATTER) =====================
\pagenumbering{roman} % Đánh số La Mã (i, ii, iii...)

% Lời cam đoan (nếu cần thì bỏ comment tạo file)
%\input{content/protestation} 

% Lời cảm ơn (nếu cần thì bỏ comment tạo file)
%\input{content/acknowledgements}

% Tóm tắt (nếu cần thì bỏ comment tạo file)
%% ===================== Content/abstract.tex =====================
\section*{Tóm tắt}
\addcontentsline{toc}{section}{Tóm tắt}

SeaWhisper là ứng dụng di động chia sẻ ẩn danh theo mô hình “thả chai” (message in a bottle), nơi người dùng có thể gửi những tâm sự của mình ra “biển” và một người khác sẽ ngẫu nhiên “nhặt” được để đọc và phản hồi một cách nhẹ nhàng. Dự án hướng đến trải nghiệm chữa lành (healing), giảm độc hại (toxic) và khuyến khích tương tác tích cực thông qua cơ chế ẩn danh có kiểm duyệt.

Mục tiêu của hệ thống là tạo ra một không gian an toàn, không phán xét để người dùng giải tỏa cảm xúc; đồng thời áp dụng bộ lọc nội dung (AI Filter) nhằm hạn chế ngôn từ công kích, kích động, hoặc tiêu cực quá mức. Về mặt kỹ thuật, ứng dụng được xây dựng với React Native cho phía client, FastAPI cho phía server, Redis làm cơ sở dữ liệu lưu trữ nhanh và cơ chế ghép ngẫu nhiên (matching) để phân phối “chai” giữa người gửi và người nhận.

Báo cáo trình bày quy trình phát triển theo hướng lặp (iterative), các quyết định thiết kế MVP, kiến trúc hệ thống, thiết kế web service, chiến lược triển khai, kiểm thử và đánh giá trải nghiệm người dùng. Kết quả khảo sát ban đầu cho thấy nhu cầu cao đối với một nền tảng ẩn danh lành mạnh; các phân tích hành vi và phản hồi người dùng được sử dụng để đề xuất cải tiến cho các phiên bản tiếp theo, bao gồm tối ưu bộ lọc, đa dạng hóa cơ chế phản hồi cảm xúc và tăng cường tính an toàn cộng đồng.


% Mục lục
\newpage
\tableofcontents

% Danh sách hình & bảng
\newpage
\listoffigures
\newpage
\listoftables

% ===================== NỘI DUNG CHÍNH (MAIN MATTER) =====================
\newpage
\pagenumbering{arabic} % Đánh số 1, 2, 3...
\setcounter{page}{1}
\setcounter{table}{0}
\setcounter{figure}{0}

% --- CÁC CHƯƠNG NỘI DUNG ---
\input{content/01_introduction}
\input{content/02_methodology}
\input{content/03_features}
\input{content/04_implementation}
\input{content/05_ux}
\input{content/06_architecture}
\input{content/07_webservice}
\input{content/08_deployment}
\input{content/09_testing}
\input{content/10_analytics}
\input{content/11_evaluation}

% ===================== PHẦN CUỐI (BACK MATTER) =====================

% Tài liệu tham khảo
\newpage
\phantomsection
\addcontentsline{toc}{chapter}{Tài liệu tham khảo}
\printbibliography

% Phụ lục (nếu có)
%\newpage
%\input{content/appendices}

\end{document}