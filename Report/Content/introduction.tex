\section{Giới thiệu}

\subsection{Bối cảnh và vấn đề}
Trong bối cảnh mạng xã hội phát triển mạnh, nhu cầu chia sẻ cảm xúc cá nhân của người trẻ ngày càng tăng.
Tuy nhiên, việc chia sẻ dưới danh tính thật thường đi kèm rủi ro bị phán xét, bị soi xét đời tư, hoặc nhận phản hồi tiêu cực.
Các nền tảng \textit{ẩn danh} giúp giảm rào cản khi bộc lộ cảm xúc, nhưng đồng thời cũng dễ trở thành môi trường độc hại do thiếu cơ chế định hướng tương tác và kiểm duyệt phù hợp \cite{anonymous_social_risks}.

Với các ứng dụng ẩn danh phổ biến, người dùng có thể gặp tình huống: nội dung bị công kích, phản hồi mang tính chế giễu, hoặc xuất hiện hành vi bắt nạt trên không gian số.
Các yếu tố này có thể tác động tiêu cực đến sức khỏe tinh thần, đặc biệt với nhóm người dùng trẻ hoặc đang ở trạng thái cảm xúc nhạy cảm \cite{online_harassment_mental_health}.
Do đó, bài toán đặt ra không chỉ là \textit{ẩn danh để dễ nói}, mà còn là \textit{thiết kế trải nghiệm để an toàn}.

\subsection{Mục tiêu dự án}
SeaWhisper được xây dựng nhằm tạo ra một không gian ẩn danh nhưng an toàn, hướng đến trải nghiệm chữa lành (healing-oriented).
Các mục tiêu chính gồm:
\begin{itemize}
  \item Cung cấp cơ chế chia sẻ ẩn danh theo mô hình “thả chai – nhặt chai” (message in a bottle), giảm áp lực khi chia sẻ.
  \item Khuyến khích tương tác đồng cảm, hạn chế tranh cãi/công kích bằng thiết kế luồng phản hồi đơn giản và có định hướng.
  \item Đảm bảo hệ thống dễ triển khai ở mức MVP nhưng vẫn có khả năng mở rộng cho các phiên bản sau.
\end{itemize}

\subsection{Phạm vi và đối tượng hướng tới}
Dự án tập trung vào nhóm người dùng trẻ (đặc biệt Gen Z), sinh viên, và những người có nhu cầu giải tỏa tâm trạng, tìm kiếm sự đồng cảm nhưng không muốn lộ danh tính.
Trong phạm vi MVP, dự án ưu tiên các chức năng cốt lõi:
\begin{itemize}
  \item Đăng ký/đăng nhập (tối thiểu để hỗ trợ lưu trữ và quản lý dữ liệu cá nhân).
  \item Thả chai (tạo thông điệp).
  \item Nhặt chai ngẫu nhiên và xem nội dung.
  \item Phản hồi đơn giản và lưu chai vào “bộ sưu tập”.
\end{itemize}

\subsection{Giá trị thực tiễn và tác động}
SeaWhisper hướng tới mô hình “C2C cảm xúc”: mỗi người vừa là người chia sẻ vừa là người lắng nghe.
Nếu triển khai thực tế, ứng dụng có thể:
\begin{itemize}
  \item Giảm áp lực tâm lý bằng việc cung cấp một kênh giãi bày ít ràng buộc hơn so với mạng xã hội danh tính thật \cite{self_disclosure_online}.
  \item Hạn chế độc hại thường thấy trong nền tảng ẩn danh bằng cách điều tiết cách tương tác và cơ chế phản hồi.
  \item Khuyến khích hành vi đồng cảm, tạo cảm giác “được lắng nghe” trong cộng đồng người dùng.
\end{itemize}

\subsection{Cấu trúc báo cáo}
Phần còn lại của báo cáo được tổ chức như sau:
\begin{itemize}
  \item Phương pháp và quy trình: mô tả cách tiếp cận lặp, tổ chức nhóm, công cụ.
  \item Tính năng: trình bày MVP và giới hạn.
  \item Triển khai và kiến trúc: mô tả công nghệ, dữ liệu, thiết kế API.
  \item Đánh giá: kiểm thử, khảo sát hài lòng và định hướng cải tiến.
\end{itemize}
