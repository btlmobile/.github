\section{Khảo sát sự hài lòng của người dùng và phân tích hành vi người dùng}

\subsection{Mục tiêu khảo sát}

Khảo sát người dùng được thực hiện nhằm đánh giá mức độ hài lòng và cảm nhận ban đầu của người dùng đối với ứng dụng SeaWhisper ở giai đoạn Minimum Viable Product (MVP). Cụ thể, khảo sát hướng tới các mục tiêu sau:

\begin{itemize}
    \item Đánh giá mức độ hài lòng tổng thể của người dùng đối với ứng dụng.
    \item Đánh giá trải nghiệm giao diện và các chức năng cốt lõi như thả chai và nhặt chai.
    \item Xác định mức độ phù hợp của ứng dụng trong việc hỗ trợ chia sẻ cảm xúc cá nhân.
    \item Thu thập ý kiến phản hồi và đề xuất cải thiện cho các phiên bản tiếp theo.
\end{itemize}

Kết quả khảo sát đóng vai trò là cơ sở quan trọng để đánh giá hiệu quả ban đầu của ý tưởng và định hướng phát triển trong tương lai.

\subsection{Phương pháp và đối tượng khảo sát}

Khảo sát được thực hiện thông qua nền tảng Google Forms với các đặc điểm sau:

\begin{itemize}
    \item Đối tượng khảo sát: Sinh viên và người trẻ trong độ tuổi từ 18 đến 24.
    \item Hình thức câu hỏi: Kết hợp câu hỏi trắc nghiệm, thang đo Likert 5 mức và câu hỏi mở.
    \item Thời điểm thực hiện: Sau khi hoàn thành phiên bản MVP của ứng dụng SeaWhisper.
\end{itemize}

Thang đo Likert 5 mức được sử dụng để đánh giá mức độ hài lòng, với các mức từ 1 (rất không hài lòng) đến 5 (rất hài lòng). Phương pháp này giúp định lượng cảm nhận người dùng một cách rõ ràng và dễ phân tích.

\subsection{Tổng quan mẫu khảo sát}

Kết quả khảo sát cho thấy đa số người tham gia thuộc nhóm sinh viên và người trẻ, phù hợp với nhóm đối tượng mục tiêu của ứng dụng SeaWhisper. Về kinh nghiệm sử dụng ứng dụng ẩn danh, phần lớn người tham gia cho biết họ đã từng sử dụng các nền tảng tương tự như NGL hoặc Whisper, cho phép họ đưa ra đánh giá dựa trên trải nghiệm thực tế.

Điều này giúp đảm bảo tính khách quan và giá trị tham khảo của kết quả khảo sát.

\subsection{Kết quả đánh giá mức độ hài lòng}

\subsubsection{Mức độ hài lòng tổng thể}

Kết quả khảo sát cho thấy khoảng 65--70\% người tham gia đánh giá mức độ hài lòng tổng thể của ứng dụng SeaWhisper ở mức hài lòng hoặc rất hài lòng. Một tỷ lệ nhỏ người dùng cho biết cảm nhận ở mức trung lập hoặc chưa hài lòng, phản ánh thực tế rằng ứng dụng vẫn đang trong giai đoạn MVP và chưa hoàn thiện đầy đủ các tính năng nâng cao.

Kết quả này cho thấy SeaWhisper đã đáp ứng tương đối tốt kỳ vọng ban đầu của người dùng về một nền tảng chia sẻ cảm xúc ẩn danh.

\subsubsection{Đánh giá giao diện người dùng}

Phần lớn người tham gia đánh giá tích cực về giao diện ứng dụng, cho rằng thiết kế mang phong cách biển và màu sắc dịu nhẹ giúp tạo cảm giác dễ chịu và thư giãn. Khoảng trên 70\% người dùng lựa chọn mức hài lòng hoặc rất hài lòng đối với tiêu chí này.

Điều này cho thấy hướng thiết kế tập trung vào trải nghiệm cảm xúc (healing-oriented design) là phù hợp với mục tiêu của ứng dụng.

\subsubsection{Đánh giá chức năng thả chai và nhặt chai}

Hai chức năng cốt lõi của SeaWhisper là thả chai và nhặt chai nhận được phản hồi tích cực từ người dùng. Đa số người tham gia cho biết các chức năng này dễ sử dụng và mang lại trải nghiệm thú vị.

Tuy nhiên, vẫn tồn tại một tỷ lệ người dùng đánh giá ở mức trung lập hoặc chưa hài lòng, chủ yếu do ứng dụng hiện chưa có các cơ chế lọc nội dung nâng cao hoặc phản hồi phong phú. Điều này được xem là phù hợp với phạm vi MVP của dự án.

\subsubsection{Mức độ phù hợp trong chia sẻ cảm xúc cá nhân}

Khoảng 70\% người tham gia cho rằng SeaWhisper là một nền tảng phù hợp để chia sẻ cảm xúc cá nhân. Kết quả này xác nhận rằng ý tưởng xây dựng một không gian chia sẻ ẩn danh mang tính chữa lành là có tiềm năng và đáp ứng nhu cầu thực tế của người dùng trẻ.

\subsection{Ý định sử dụng trong tương lai}

Khi được hỏi về ý định sử dụng SeaWhisper trong tương lai, phần lớn người tham gia cho biết họ sẵn sàng hoặc có thể tiếp tục sử dụng ứng dụng nếu được phát triển và hoàn thiện thêm. Tổng tỷ lệ người dùng lựa chọn “Có” hoặc “Có thể” chiếm trên 80\%, trong khi một tỷ lệ nhỏ cho biết họ không có ý định sử dụng tiếp.

Kết quả này cho thấy ứng dụng có tiềm năng phát triển người dùng nếu được cải thiện thêm về tính năng và trải nghiệm.

\subsection{Phân tích hành vi người dùng (User Behavior Analytics)}

Do dự án đang ở giai đoạn MVP và chưa được triển khai rộng rãi, nhóm chưa tích hợp các công cụ phân tích hành vi người dùng tự động như Google Analytics hoặc Firebase Analytics. Thay vào đó, phân tích hành vi người dùng được thực hiện dựa trên các phương pháp định tính, bao gồm:

\begin{itemize}
    \item Quan sát quá trình sử dụng ứng dụng trong các buổi kiểm thử nội bộ.
    \item Phản hồi trực tiếp từ người dùng tham gia khảo sát.
    \item Đánh giá khả năng hoàn thành các luồng chức năng chính.
\end{itemize}

Kết quả cho thấy người dùng có thể dễ dàng tiếp cận và hoàn thành các luồng chính như đăng nhập, thả chai, nhặt chai và lưu bottle. Các phản hồi định tính cho thấy người dùng quan tâm nhiều nhất đến trải nghiệm đọc thông điệp và cảm giác đồng cảm mà ứng dụng mang lại.

\subsection{Tổng hợp phản hồi và đề xuất cải thiện}

Phân tích các câu trả lời mở cho thấy một số đề xuất cải thiện phổ biến bao gồm:

\begin{itemize}
    \item Bổ sung thêm các hình thức phản hồi cảm xúc đa dạng hơn.
    \item Cải thiện khả năng phân loại hoặc gợi ý nội dung.
    \item Hoàn thiện thêm giao diện và hiệu ứng tương tác.
\end{itemize}

Những đề xuất này được xem là định hướng quan trọng cho các phiên bản phát triển tiếp theo của SeaWhisper.

\subsection{Dữ liệu khảo sát}

Dữ liệu khảo sát chi tiết được tổng hợp và lưu trữ dưới dạng bảng tính Google Sheets nhằm phục vụ việc đối chiếu và kiểm tra khi cần thiết. Liên kết dữ liệu khảo sát được công bố tại:

\begin{center}
\url{https://docs.google.com/spreadsheets/d/1OytnulUoMh47sXLI4KIllON-HdLEMNFof2FuMkmmqxE/edit}
\end{center}

Trong báo cáo này, nhóm chỉ trình bày các kết quả tổng hợp và phân tích chính nhằm đảm bảo tính súc tích và tập trung vào mục tiêu đánh giá trải nghiệm người dùng.
