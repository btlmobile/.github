\section{Tính năng và chức năng hệ thống}

\subsection{Tổng quan các tính năng}
SeaWhisper được thiết kế xoay quanh trải nghiệm chia sẻ ẩn danh mang tính chữa lành.
Trong MVP, hệ thống tập trung vào các luồng cốt lõi: thả chai, nhặt chai, phản hồi và lưu trữ.
Thiết kế này bám sát ý tưởng “message in a bottle”, tạo yếu tố ngẫu nhiên và kết nối gián tiếp giữa người dùng.

\subsection{Gửi chai (Send Bottle)}
Người dùng có thể tạo một “chai” bằng cách nhập nội dung tâm sự và gửi lên hệ thống.
Về mặt trải nghiệm, màn hình gửi được tối giản để người dùng tập trung vào việc viết.

Về mặt hệ thống:
\begin{itemize}
  \item Client gửi request tạo bottle.
  \item Backend validate dữ liệu (rỗng/quá dài).
  \item Bottle được lưu vào Redis với định danh và timestamp.
\end{itemize}

\subsection{Nhận chai ngẫu nhiên (Receive Bottle)}
Khi người dùng nhặt chai, hệ thống trả về ngẫu nhiên một bottle đang khả dụng.
Tính “ngẫu nhiên có kiểm soát” giúp tăng cảm giác khám phá, đồng thời tránh lặp chai nhiều lần trong cùng ngữ cảnh xử lý.

Trường hợp không có bottle phù hợp, backend trả về 204 No Content để client hiển thị trạng thái “biển đang vắng”.

\subsection{Phản hồi (Chat/Reply)}
Sau khi đọc nội dung, người dùng có thể phản hồi bằng tin nhắn ngắn trong giao diện chat tối giản.
Trong MVP, cơ chế phản hồi được thiết kế để hạn chế toxic bằng việc:
\begin{itemize}
  \item Giới hạn độ dài phản hồi (nếu có).
  \item UI tập trung vào động viên/đồng cảm.
\end{itemize}

\subsection{Lưu trữ và xem lại chai}
Người dùng có thể lưu bottle vào “balo lưu trữ” và xem danh sách đã lưu theo trang.
Mục tiêu là tăng tính gắn bó: người dùng có thể quay lại đọc các thông điệp có ý nghĩa.

\subsection{Giới hạn của MVP}
MVP chưa triển khai:
\begin{itemize}
  \item Kiểm duyệt nội dung nâng cao (moderation) ở mức production.
  \item Analytics tự động (Firebase/GA).
  \item Cá nhân hóa theo chủ đề/hành vi.
\end{itemize}
Các mục này được trình bày ở phần định hướng cải tiến.
