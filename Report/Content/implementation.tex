\section{Triển khai hệ thống}

\subsection{Tổng quan}
SeaWhisper triển khai theo mô hình client--server.
Client là ứng dụng di động React Native (Expo), backend là FastAPI cung cấp REST API, và Redis làm nơi lưu trữ chính ở mức MVP.
Thiết kế này giúp tách biệt giao diện và nghiệp vụ, thuận tiện mở rộng và bảo trì.

\subsection{Frontend với React Native Expo}
Frontend được phát triển bằng React Native và Expo để rút ngắn thời gian cấu hình môi trường và tăng tốc thử nghiệm trên thiết bị thật \cite{react_native_docs, expo_docs}.
Ứng dụng tổ chức theo hướng component-based:
\begin{itemize}
  \item Screen components: Home, Send, Receive/Detail, Chat, Stored list, Account/Setting, Introduce/Support.
  \item Shared components: button, modal, input, loading overlay.
\end{itemize}

Giao tiếp backend qua HTTP, dữ liệu JSON.
Token (nếu có) được lưu ở storage phía client và đính kèm header Authorization cho các API cần quyền.

\subsection{Điều hướng và quản lý trạng thái}
Điều hướng được thiết kế ưu tiên luồng chính (thả/nhặt).
State ở mức MVP có thể quản lý bằng state cục bộ theo screen và context cho session (user/token/theme).
Các trạng thái quan trọng:
\begin{itemize}
  \item Loading / error khi gọi API.
  \item Session: token, user id.
  \item Theme: light/night cho các màn hình hỗ trợ.
\end{itemize}

\subsection{Backend với FastAPI}
Backend dùng FastAPI vì hỗ trợ kiểu dữ liệu rõ ràng, tốc độ phát triển nhanh và tự động sinh OpenAPI/Swagger \cite{fastapi_docs, openapi_spec}.
Backend tách lớp:
\begin{itemize}
  \item Router: định nghĩa endpoint.
  \item Service: xử lý nghiệp vụ (auth, bottle, stored bottle).
  \item Data access: thao tác Redis.
  \item Schema/DTO: chuẩn hóa request/response.
\end{itemize}

\subsection{Redis làm kho lưu trữ MVP}
Redis phù hợp cho MVP vì truy xuất nhanh và hỗ trợ cấu trúc dữ liệu đơn giản \cite{redis_docs}.
Các key có thể tổ chức theo namespace:
\begin{itemize}
  \item user:{id}
  \item bottle:{id}
  \item stored:{user\_id} (list/set các stored bottle id)
\end{itemize}

\subsection{Cơ chế phân phối ngẫu nhiên}
Khi nhặt chai, backend chọn một bottle khả dụng (chưa bị đánh dấu đã phát) theo cơ chế đơn giản.
Sau khi trả về, hệ thống cập nhật trạng thái để hạn chế phân phối trùng lặp trong cùng ngữ cảnh xử lý.
Ở phiên bản mở rộng, cơ chế random có thể chuyển sang hàng đợi, ưu tiên theo thời gian, hoặc theo tag/chủ đề.

\subsection{Giới hạn triển khai}
Do tập trung MVP, hệ thống chưa tối ưu:
\begin{itemize}
  \item Quan sát/giám sát (monitoring) và logging chuẩn production.
  \item Test tự động đầy đủ.
  \item Cơ chế chống spam và moderation hoàn chỉnh.
\end{itemize}
