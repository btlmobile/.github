% ===================== Content/methodology.tex =====================
\section{Phương pháp và quy trình thực hiện}

\subsection{Cách tiếp cận phát triển}
Dự án SeaWhisper được phát triển theo hướng tiếp cận lặp (iterative development), tập trung vào việc xây dựng và hoàn thiện các chức năng cốt lõi của MVP trước khi mở rộng thêm các tính năng nâng cao. Nhóm ưu tiên triển khai những thành phần mang lại giá trị trực tiếp cho người dùng trong thời gian giới hạn của học kỳ.

Thay vì xây dựng một hệ thống phức tạp ngay từ đầu, dự án tập trung vào việc hiện thực hóa ý tưởng chính là mô hình “thả chai – nhặt chai”, đảm bảo luồng sử dụng đơn giản, ổn định và dễ mở rộng trong tương lai.

\subsection{Quy trình làm việc nhóm}
Nhóm áp dụng quy trình làm việc theo các bước sau:
\begin{itemize}
    \item Phân tích yêu cầu và xác định phạm vi MVP.
    \item Thiết kế luồng người dùng (user flow) và kiến trúc tổng thể.
    \item Phân chia nhiệm vụ theo vai trò Frontend và Backend.
    \item Triển khai từng chức năng độc lập, sau đó tích hợp hệ thống.
    \item Kiểm thử và tinh chỉnh dựa trên trải nghiệm sử dụng.
\end{itemize}

Các nhiệm vụ và tiến độ được quản lý thông qua GitHub Project Board, cho phép theo dõi trạng thái công việc và phối hợp giữa các thành viên một cách hiệu quả.

\subsection{Công cụ và nền tảng sử dụng}
Trong quá trình phát triển, nhóm sử dụng các công cụ sau:
\begin{itemize}
    \item \textbf{GitHub}: quản lý mã nguồn, phân chia nhánh và theo dõi tiến độ.
    \item \textbf{Figma}: thiết kế giao diện và mô phỏng luồng người dùng.
    \item \textbf{Visual Studio Code}: môi trường phát triển chính cho cả frontend và backend.
    \item \textbf{Postman}: kiểm thử và xác minh các API backend.
\end{itemize}

\subsection{Quyết định phạm vi MVP}
Do giới hạn về thời gian và nguồn lực, nhóm quyết định tập trung vào các chức năng thiết yếu nhất của hệ thống, bao gồm:
\begin{itemize}
    \item Gửi nội dung ẩn danh dưới dạng “chai”.
    \item Nhận ngẫu nhiên “chai” từ người dùng khác.
    \item Phản hồi cảm xúc cơ bản và lưu trữ lịch sử tương tác.
\end{itemize}

Các chức năng nâng cao như phân tích cảm xúc tự động hoặc bộ lọc thông minh được xác định là hướng phát triển trong các phiên bản tiếp theo, nhưng không nằm trong phạm vi triển khai của MVP.

\subsection{Đánh giá phương pháp}
Cách tiếp cận tập trung vào MVP giúp nhóm:
\begin{itemize}
    \item Hoàn thành được sản phẩm khả dụng trong thời gian học kỳ.
    \item Giảm độ phức tạp trong quá trình triển khai và kiểm thử.
    \item Dễ dàng mở rộng hệ thống trong tương lai khi có thêm thời gian và tài nguyên.
\end{itemize}
