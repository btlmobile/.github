\section{Phương pháp và quy trình thực hiện}

\subsection{Cách tiếp cận phát triển}
Dự án SeaWhisper được phát triển theo hướng lặp (\textit{iterative development}).
Thay vì cố gắng hoàn thiện đầy đủ ngay từ đầu, nhóm ưu tiên xây dựng nhanh một phiên bản khả dụng (MVP) để:
(1) kiểm chứng ý tưởng “thả chai – nhặt chai”, (2) đánh giá trải nghiệm người dùng, và (3) giảm rủi ro kỹ thuật trong thời gian giới hạn học kỳ \cite{mvp_concept}.

Với cách tiếp cận này, mỗi vòng lặp gồm: đặc tả yêu cầu nhỏ $\rightarrow$ thiết kế UI/flow $\rightarrow$ triển khai $\rightarrow$ kiểm thử thủ công $\rightarrow$ cải tiến.

\subsection{Quy trình làm việc nhóm}
Nhóm tổ chức công việc theo các bước:
\begin{itemize}
  \item Phân tích yêu cầu và chốt phạm vi MVP.
  \item Thiết kế UI trên Figma, thống nhất luồng thao tác.
  \item Chia task Frontend/Backend theo module (Auth, Bottle, StoredBottle, UI screens).
  \item Tích hợp API và hoàn thiện luồng end-to-end.
  \item Kiểm thử bằng Swagger UI/Postman và chạy thử trên thiết bị thật.
\end{itemize}

Mã nguồn và cộng tác nhóm được quản lý qua GitHub (branching, pull request, review), giúp theo dõi lịch sử thay đổi và phân chia trách nhiệm rõ ràng \cite{github_flow}.

\subsection{Công cụ và nền tảng sử dụng}
Các công cụ chính:
\begin{itemize}
  \item GitHub: quản lý mã nguồn, issues, pull requests \cite{github_docs}.
  \item Figma: thiết kế UI và prototype.
  \item React Native + Expo: phát triển ứng dụng di động đa nền tảng \cite{react_native_docs, expo_docs}.
  \item FastAPI: triển khai REST API, tự sinh tài liệu OpenAPI/Swagger \cite{fastapi_docs, openapi_spec}.
  \item Redis: lưu trữ dữ liệu nhanh cho MVP \cite{redis_docs}.
  \item Swagger UI/Postman: kiểm thử API thủ công \cite{swagger_docs, postman_docs}.
\end{itemize}

\subsection{Quyết định phạm vi MVP}
Do giới hạn thời gian, nhóm tập trung vào các chức năng thiết yếu:
\begin{itemize}
  \item Auth cơ bản (đăng ký/đăng nhập) để phục vụ lưu trữ “bộ sưu tập”.
  \item Tạo bottle và lấy bottle ngẫu nhiên.
  \item Lưu bottle, xem danh sách đã lưu, xóa bottle.
\end{itemize}

Các hạng mục nâng cao như phân tích cảm xúc tự động, lọc nội dung nâng cao, cá nhân hóa feed được đưa vào định hướng tương lai.

\subsection{Đánh giá phương pháp}
Cách làm tập trung MVP giúp nhóm:
\begin{itemize}
  \item Hoàn thành sản phẩm end-to-end trong học kỳ.
  \item Giảm độ phức tạp thiết kế dữ liệu và hạ tầng.
  \item Có cơ sở đo lường/đánh giá (survey) để đề xuất cải tiến có căn cứ.
\end{itemize}
